\newchapter{Approaching Vivec}{Anonymous}

\cnote{Who is Vivec?}
\dropcap Morrowind is holy country, and its gods are flesh and blood. Collectively, these gods are called the Tribunal, three deities exemplifying Dunmeri virtues. Almalexia is Mercy, Vivec is Mastery, and Sotha Sil is Mystery. Vivec is easiest the most popular of them all.

He is also the most public, for he is the beloved Warrior-Poet of the True People, paradoxically beautiful and bloody. Vivec is an artistic violence. He is transcendent of the Dark Elven demon that anticipated him, Black Hands Mephala, a foundation figure of the earliest Chimer. Modern Vivec mirrors the ur-Mephala. We shall take them hand in hand.

\cnote{Who is Mephala?}
Each of the three Tribunes were present at the dawn of Chimeri culture, at least in spirit. According to legend, three demons, or Daedra, helped a discontented throng of Altmer become a new people and found a new land. These Altmer became the Chimer, or “the Changed Folk”. This was more than an ideological shift or political statement; the Chimer \textit{physically} changed as well. Details of this transformation can be found elsewhere, but each of the three demons represented a crucial part of its metaphysics. If Boethiah, the so-called Prince of Plots, represented the method needed to bring about change, Mephala was the shadowy enforcer of that scheme.

Mephala is the demon of murder, sex, and secrets. All of these possess subtle aspects and violent ones (assassination/ genocide, courtship/orgy, tact/ poetic truths); Mephala was meant to embody those dichotomies, and this made it (Mephala is hermaphroditic) a difficult deity to understand. It is no surprise that Vivec exploits the more popular characteristics of his progenitor: combat and art.

Mephala has both male and female genitalia, and both are grossly exaggerated in the idols, drawings, and carvings that depict it. Androgyny is sometimes depicted in Vivec as well, but not as overtly. He (notice the pronoun) is almost always represented as a male, though often with homosexual or bisexual tendencies.

As has been said, reverence of Mephala was co-opted into the worship of Vivec. Legends and myths attributed to the demon now serve as a relief to the god to come later. This is not to say that Mephala has entirely disappeared from contemporary Dunmeri worship; it has not, and survives to small extent in various thuggish mystery cults, sexual specialists, covert fashion clubs, and elsewhere. Mephala is most famous as the psychopomp of the Morag Tong, the elite assassins guild of Morrowind.