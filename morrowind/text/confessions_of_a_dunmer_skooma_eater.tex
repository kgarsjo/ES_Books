\newchapter{Confessions of a Dunmer Skooma Eater}{Anonymous}

\dropcap Nothing is more revolting to Dunmer feeling than the sorry spectacle of another Dunmer enslaved by that derivative moon-sugar known as "skooma." And nothing is less appetising than listening to the pathetic tales of humiliation and degradation associated with a victim of this addictive drug.

Why, then, do I force myself upon you with this extended and detailed account of my sins and sorrows?

Because I hope that by telling my tale, the hope of redemption from this sorry state shall be more widely known. And because I hope that others who have also fallen into the sorry state of skooma addiction may therefore hear of my story, of how I fell into despair, and how I once again found myself and freed myself from my own self-imposed chains.

Because it is widely known to all Khajiit, who may be expected to know, that there is no cure for addiction to skooma, that once a slave to skooma, always a slave to skooma. Because this is widely known, it is taken to be true. But it is not true, and I am living proof.

There is no miracle cure. There is no potion to be taken. There is no magical incantation which frees you from the thrill of skooma running through your blood.

But it is through the understanding of that thrill, and the acceptance of the lust within oneself for that thrill, and the casting aside of the shame that the thrillseeker feels when he cannot set aside what becomes in the end his only comfort and pleasure, it is through this knowledge and understanding that the victim comes to the place where choices may be made, where despair and hope may be separated.

In short, only knowledge and acceptance can deliver into the slave's hands the key that opens his shackles and sets him free.

[The narrative of Tilse Sendas' tale carries the reader through the stages of early infatuation, ecstatic obsession, and profound degradation of her addiction, and in the course of the story she subtly enables the reader to discover that the hopelessness of the addict comes from the addict's own unconscious assumption that only a helpless and foolish person could become addicted to skooma, and that, consequently, no such helpless and foolish person could ever achieve the admittedly difficult task of renouncing, once tasted, the exquisite delights of the skooma. Tilse Sendas shows that once the addict overcomes the burden of her own self-despising, that there is the possibility of redemption. And, against all of society's dearly held beliefs, she says that it is not altogether clear that the addict SHOULD renounce the sugar, but that it is only one of the choices that the skooma addict must make. Tilse Sendas' casual proposition that skooma addiction is not necessarily a sign of moral and personal weakness is essential to her thesis that a cure is possible, but it has not endeared her or her book to the upright and conservative elements of Dunmer society.]