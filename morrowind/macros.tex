% Section line - Sets a fancy horizontal rule for rigid prose breaks
\newcommand{\sectionline}[1][black]{%
	\nointerlineskip \vspace{.3\baselineskip}\hspace{\fill}
	{\resizebox{0.85\linewidth}{0.3ex}
 		{\pgfornament[color = #1]{89}
    }}%
	\hspace{\fill}
	\par\nointerlineskip \vspace{.5\baselineskip}
}

% Blank Page

% Chapter Heading with title and author
\newcommand{\cchapter}[2]{%
	\chapter*{\headerfont #1\newline\small By #2}%
	\addcontentsline{toc}{chapter}{#1}}

% Section Heading
\newcommand{\csection}[2][Chapter]{%
	\section*{\headerfont #1 #2}%
	\addcontentsline{toc}{section}{#1 #2}}

% Part Heading
\newcommand{\cpart}[1]{%
	\part*{\headerfont #1}%
	\addcontentsline{toc}{part}{#1}}

\newcommand{\cnote}[1]{\small\textit{#1}\newline}

% Para Break - Sets a blank line for soft prose breaks between paragraphs (preserves indentation)
\newcommand{\parabreak}[0]{\sectionline[white]}

% Drop Cap - Uniform method of accenting a paragraph's beginning
\newcommand{\dropcap}[3][6]{\lettrine[lines=#1]{{\blockfont\color{red}#2}}{#3}}

% Quote
\newcommand{\cquote}[3]{%
	\begin{quotation}
	\small#1
	\sourceatright{\small---#2}
	\end{quotation}
}