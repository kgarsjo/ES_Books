% Section line - Sets a fancy horizontal rule for rigid prose breaks
\newcommand{\sectionline}[1][black]{%
	\nointerlineskip \vspace{.3\baselineskip}\hspace{\fill}
	{\resizebox{0.85\linewidth}{0.3ex}
 		{\pgfornament[color = #1]{89}
    }}%
	\hspace{\fill}
	\par\nointerlineskip \vspace{.5\baselineskip}
}

% newchapter - Starts a new chapter with the given content
\newcommand{\newchapter}[2]{
	\chapter{\headerfont #1\newline\small By #2}
}

% dropcap - Formats the first given word as a drop-cap
\newcommand{\dropcap}[2][6]{
	\lettrine[lines=#1]{\blockfont\color{red}\strcar{#2}}{\strcdr{#2}}
}

% strcar - Splits the string to return the first character
\newcommand{\strcar}[1]{
	\StrLeft{#1}{1}
}

% strcdr - Splits the string to return the characters following the first
\newcommand{\strcdr}[1]{
	\StrGobbleLeft{#1}{1}
}

% Section Heading
\newcommand{\csection}[2][Chapter]{%
	\section*{\headerfont #1 #2}%
}

% Part Heading
\newcommand{\cpart}[1]{%
	\part{\headerfont #1}%
}

\newcommand{\cnote}[1]{
	\subsection*{\small\textit{#1}}
}

% Para Break - Sets a blank line for soft prose breaks between paragraphs (preserves indentation)
\newcommand{\parabreak}[0]{\sectionline[white]}

% Quote
\newcommand{\cquote}[3]{%
	\begin{quotation}
	\small#1
	\sourceatright{\small---#2}
	\end{quotation}
}