\cchapter{The Fall of the Usuper}{Palaux Illthre}

\dropcap{T}{he} people of Dwynnen celebrate Othroktide every 5th of Suns Dawn, the date when, according to legend, a man emerged from the wilderness of High Rock and defeated the undead of Castle Wightmoor to become the first Baron of Dwynnen. Few people believe the legend anymore, but there most certainly was a Baron Othrok of Dwynnen who was destined to become one of true heroes of High Rock, if not all Tamriel.

The legend, as most any Dwynnen child will tell you, is that years and years ago (archivists have agreed to the year 3E253), the people of Dwynnen were ruled by a lich and its armies of zombies, ghosts, vampires, and skeletons. Othrok was blessed with by gods and given an army of men and animals to destroy the dead. He brought peace and prosperity to the land, growing more powerful as the land improved. Years later, he led the tiny barony against the Camoran Usurper, and saved all of Tamriel.

How much credit the Baron ought to receive for the defeat of the Camoran Usurper has been debated, but it is an uncontestable fact that in the year 3E 267, the Camoran Usurper's relentless move north through High Rock was halted around the area of contemporary Dwynnen. Dwynnen is actually larger than it was in the first Baron's day -- it did not, in fact, have a sea port -- but the Battle of Firewaves was a coastal battle. The fact that the battle probably did not occur in Dwynnen does not in itself belittle the Baron's participation in it.

The Camoran Usurper had conquered Hammerfell and Valenwood by means of a large army, which by legend consisted entirely of undead and daedra, but was mostly composed of Redguards and Wood Elves. In all probability, the Usurper summoned the daedra and undead in Arenthia and slowly replaced the original summoned creatures with the armies of his conquered territories. Most armies of Valenwood have been historically mercenary.

Word of the Usurper's conquests reached High Rock in early 266, but preparations to repel the invasion did not begin until early the following year. Historians attribute two factors to High Rock's hesistancy. The primary powers of the Bay were ruled by particularly inept monarchs -- Wayrest and Sentinel both had kings in their minority, and Daggerfall was torn by contention between Helena and her cousin Jilathe. The Lord of Reich Gradkeep (now Anticlere) was deathly ill through 266 and finally died at the end of the year. There were, in short, no leaders to unite the province against the Usurper. Of the leaders with any influence, at least eight (the "Eight Traitors" of legend) made secret allegiances with the Usurper to protect their lands.

The secondary reason for the lethargy of High Rock had to do with the depth of relations between the province and the Septim Empire. For the first time since the beginning of the Dynasty, an Emperor ruled Tamriel who was neither Breton nor had spent any of his childhood in High Rock. The difference between Cephorus II and his cousin Uriel IV who preceded him was appalling to the people of High Rock. Even mad Emperors like Pelagius III revered the Bretons over all other races, and cousins and younger siblings of the Emperors have ruled in High Rock since the foundation of the Empire. Cephorus was a Nord, with Skyrim and Morrowind sympathies. The attitude of the common men of High Rock was sympathetic toward the Camoran Usurper as an archfoe of this hated Emperor.

The Baron and his less legendary allies, the rulers of Ykalon, Phrygia, and Kambria, changed this favorable perception. News of the Usurper's barbaric treatment of captives and abuse of conquered lands, mostly true, spread rapidly through their territories, and then to other neutral lands. Within a few months, the greatest navy ever combined organized along the High Rock edge of the Iliac Bay. Only the navy of Uriel V's illfated invasion of Akavir was comparable.

How the combined forces of High Rock defeated the endless army of the Camoran Usurper is certainly worthy of a lengthy book in itself. And perhaps, it is best left to the public imagination. Certainly the weather worked against the Usurper, which is reason in itself to attribute divine intervention.

Baron Othrok's divine purpose is the central theme to Othroktide, after all. And as the poet Braeloque wrote, "To find the facts, the wisest always look first to the fiction."