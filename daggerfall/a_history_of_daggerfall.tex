\cchapter{A History of Daggerfall}{Odiva Gallwood}

\dropcap{T}{here} is sufficient archaeological evidence for the modern historian to believe that there has been some variety of human settlement in the city-state of Daggerfall starting at least a thousand years before recorded history. The first use of the name Daggerfall to refer to the area around the current capitol was most probably in the 246th year of the 1st Era. The north half of the Iliac Bay, in fact all of the current province of High Rock, was conquered by invading Nords who brought a rough sort of civilization with them. One of the first civilized acts the Nords performed was a census -- the so-called Book of Life. Listed on page 933 of the Book is this entry:

"North of the Highest bluffs, south of the moors, west of the hills, and east of the sea is called DAGGERFALL. 110 men, 93 women, 13 children under 8 years of age, 58 cows, 7 bulls, 63 chickens, 11 cocks, 38 hogs live here."

Nearly four thousand years after this census was taken, we can see that these two hundred and sixteen people have multiplied heartily. The last census, in the year 3E 401, lists the population at over 110,000. It is always difficult to find an exact number, but the capitol city of Daggerfall certainly outnumbers her rivals, Sentinel and Wayrest.

It has been a consistant, if not actually helpful amusement of historians to find the origin of placenames. Daggerfall, by tradition, is said to refer to the knife the first chieftain threw to form the borders of his lands. But there are other legends that may have equal validity.

The Daggerfall entry from the Book of Life actually supports one theory about the reason for Daggerfall's longevity. The people were coastal fishermen, but they also found the land itself sufficiently rich to support raising livestock. This inclination of the early citizenry toward reinforcing their principal products brought stability to a fickle land.

Daggerfall thrived during the years of the Skyrim occupancy. When the Wild Hunt killed King Borgas of Winterhold in 1E 369, the northlands engaged in the War of Succession and Skyrim, greatly weakened, lost her holdings in High Rock and Morrowind. The Iliac Bay had become important strategically, and Daggerfall began to expand her military.

There were multiple opportunities for her to exercise this army and navy during the Direnni conflicts with the force of the Alessian Reform. The Dirennis were native Bretons, and Bretons are hardly ever given to excessive religion. Daggerfall became a minor base of operations for the Dirennis and their allies. Raven Direnni, the enchantress whose magic helped secure the final victory over the Alessians in the Glenumbria Moors, was one of the earliest occupants of Castle Daggerfall.

Over the centuries that followed, the Dirennis felt into obscurity, but Daggerfall continued her growth. In 1E 609, King Thagore of Daggerfall defeated the army of Glenpoint and became the preeminant economic, cultural, and military force in southern High Rock. A position the kingdom has precariously kept ever since.

Ironically, it was another successful military exercise three hundred and seventy years later that ended Daggerfall's monopoly of Bay trade: the annihilation of the orcish capitol Orsinium by a joint effort of Daggerfall, the new kingdom of Sentinel, and the now extinct Order of Diagna. The scattering of the orcs from southeastern High Rock made the river route to the Bay more accessible. The tiny village of Wayrest grew like a flower that no longer feared the mow. In twenty years, Wayrest's trade profits equalled Daggerfall's. In forty years, Wayrest was the acknowledged master of Iliac Bay trade. In one hundred and twenty years, Wayrest became the Kingdom of Wayrest.

The Kingdom of Sentinel did not exhibit Wayrest's aggrandizement during the First Era. The Redguards were warriors learning the ways of the merchants, and their land was enemy enough to keep their population checked. Indeed, the number of people in all areas of the Iliac Bay was halved once in the First Era by the Thrassian Plague, once again by the War of Righteousness, and a third time by the invading Akavari. If Daggerfall had not spent its first thousand years preparing for the battles of the next thousand years, it is indeed conceivable that the Iliac Bay today might be Akavarian.

The Second Era, like the latter part of the First Era, is a tapestry of wars, insurrections, and plagues. Daggerfall, Sentinel, and Wayrest continued to expand and improve their military and economic positions in the Bay. Daggerfall and Wayrest would transpose positions as major trading center of the Bay, and Daggerfall and Sentinel likewise bandied over which was the superior military power.

The Iliac Bay has continued to hold an important position in the Imperial government of the Third Era. Rarely allies (though the combined armies in opposition to the Camoran Usurper in the 3rd century of the 3rd Era is a notable exception), but not always enemies, Daggerfall, Sentinel, and Wayrest have weathered the storms of contention, plague, famine, and pestilence. The recent War of Betony was typical of Iliac Bay warfare: sincere, frighteningly violent, and peaceably resolved.