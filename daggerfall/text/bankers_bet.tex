\cchapter{The Banker's Bet}{Porbert Lyttumly}

\dropcap{I}{t} was a perfectly ordinary day at the main office of the Bank of Daggerfall. Normal transactions took place: deposits were deposited, withdrawals were withdrawn, house mortgages were collected, letters of credit were golded. When a teller named Clyton J. Wifflington saw the little old lady approaching him, dragging two large sacks, each nearly as large as her, he changed his mind. It was not to be a perfectly ordinary day at the Bank of Daggerfall after all.

"I would like you to take the thirty million gold pieces I have in these sacks and open me an account," croaked the little old biddy.

"Certainly, madam," Wifflington said, eagerly. He counted the gold in the sacks and found that it was thirty million gold exactly.

"One moment, sonny," the little old lady chirruped. "Before I open the account, I would like to meet the man I'm trusting it to. I'd like to talk to the president of the bank."

Wifflington wanted the president to know that he was the teller who had taken the largest single deposit that year, so eagerly sent word to the president's secretary. As it turned out, the president was equally eager to meet such a wealthy woman, so the old lady was brought to his office that very day.

"Pleased to make your acquaintance, milady. I am Gerander P. Baggledon," said the president, Gerander P. Baggledon.

"My name," said the little old lady. "Is Petuva Smuthworthy." That was, in fact, her real name. "Thank you for seeing me. I like to conduct my business in a more personal way."

"I can certainly appreciate that," said Baggledon chucklingly. "It is an appreciable sum of gold. Would it be rude of me to ask how you came by it?"

"Not at all," said Mrs. Smuthworthy.

"How came you by it?" asked Baggledon.

"I'll let you guess," replied Mrs. Smuthworthy, with a trace of unattractive girlish flirtation.

Baggledon was a man of enormous imagination, for a banker. He guessed inheritance and longtime thrift, but Mrs. Smuthworthy coyly shook her head. Perhaps she had sold a large, old mansion? No. In a moment of chumminess, Baggledon asked if the gold came as a result of plunder or thievery. Mrs. Smuthworthy took no offense, but said no. Finally, he admitted defeat.

"I'm a gambler," she said.

"In arena fights?" he asked, interested.

"No, no, dearie. Different things. For example, I'd be willing to wager twenty five thousand gold pieces that at this time tomorrow morning, your testicles will be covered with feathers."

Mr. Baggledon was somewhat taken aback by the old woman's words. Could she be mad? Could she be a witch? He eliminated the latter possibility, for he had a sense for such things. If she were mad, she was still a rich madwoman. And he could use twenty five thousand gold pieces. So he took her wager.

For the next twenty-four hours, Mr. Baggledon obsessed over his testicles. He checked his pants so often that afternoon, his subordinates feared the worse and suggested that he not touch anything and go home for the rest of the afternoon. He spent the night seated, his pants around his ankles, his beady banker's eyes focused on his scrotum. Every time he started to doze off, his vision was filled with images of Mrs. Smuthworthy plucking feathers from his balls, cackling.

Mr. Baggledon arrived at the bank late the next day -- only moments before Mrs. Smethworthy's arrival. Accompanying her was a lean, bespeckled fellow she introduced as a barrister from the court. Her son, it turned out. Young Mr. Smethworthy always accompanied his mother when there was money involved, she explained.

"Enough banter," she crowed. "Our bet, dearie?"

"My dear, dear madam, I can tell you that your gold will be quite safe at the Bank of Daggerfall. I hope it will not cause you distress to discover that your gold will be safer here than in your own hands. My family jewels are quite, shall we say, featherless. And you owe me a sum equally twenty five thousand gold."

Poor Mrs. Smethworthy's face fell when she heard this. "Are you sure?"

"Quite, madam."

"Not even one feather?" Her voice suggested doubt. Mr. Baggledon could tell she thought he might be lying.

"Not one, I fear, madam."

"It's not that I don't trust you, Mr. Baggledon, but it is quite a lot of gold. Might I -- would you -- could I possibly see for myself?"

As he knew he was soon to be a twenty five thousand gold pieces richer, and he was still a bit punchy from lack of sleep, Mr. Baggledon merely smiled and dropped his breeches to the floor. Mrs. Smethworthy examined his testicles very carefully, under, to the left, to the right. At last, she was satisfied that there was not so much as a down feather anywhere in the region. While she was looking under them one last time, Mr. Baggledon heard a thwacking noise across the office. Young Mr. Smethworthy was banging his head against the stone wall.

"What in the Lady's name is wrong with your son, Mrs. Smethworthy?" he asked.

"Nothing, dear," she said. "I merely bet him one hundred thousand gold pieces that by this time I would have the president of the Bank of Daggerfall by the balls."