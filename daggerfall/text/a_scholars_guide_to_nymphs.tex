\cchapter{A Scholar's Guide to Nymphs}{Vondham Barnes}

\dropcap{I}{} grew up a scholar, an ascetic devoted to knowledge, with eyes that saw beauty in a fascinating passage in a dusty tome, love in the candle that allowed me to study on starless nights, passion in a well-reasoned argument of a long dead issue. I was a student who never graduated and was never expelled.

Though I am not defending myself, I should further define myself. I am not what you would call a prude. In fact, I can speak of subjects in a detached way that would make the most debauched strumpet in Skyhawk blush with discovered modesty. I wrote an essay the House of Dibella as a scholar should, analysing the cult of beauty and physical relations as one might study crop rotation or the digestive system of an orc. The acquaintances of mine who were inclined to wink and giggle I tolerated, but barely.

With all that said, the reader will understand that when I decided to study the language of the nymphs in order to study their character and culture, it was not a decision I made on account of prurience or lust. Scholars have historically neglected the nymph as a subject worthy of research, and this neglect I attribute to prejudice. The sages with whom I have spoken on the subject have eloquently and intelligently formed sentences which, boiled down, can be translated as: "Nymphs look like beautiful, naked women who skip along tra-la-la and like to have indisciminate sex. What could they have to say that would be of any interest?"

So here I was faced with the most daunting of projects -- to study and research a species unstudied is a potentially rewarding challenge. If the subject was unstudied because the scientific community had deemed it beneath interest, a potentially rewarding but decidedly frustrating challenge. If I spent months in serious study of their language and culture and additional time in their company, and discovered nothing more than that the common prejudice is correct, the term "laughing stock" would not do me justice.

So, excited and nervous for reasons unrelated to the notoriously promiscuous behavior of my subjects, I began my studies. I mastered the language, a melodious tongue that sounds like wild elf and faerie but share no vocabulary with them. I studied the lore, and found it to be on the whole, little more than pornography and crude conjecture.

I next had to find a nymph.

From my centralized location in the Imperial City, I found it easy to send word around to several wellknown temples and guilds devoted to study in all the provinces. Not all replies back were serious in nature, but one, from the School of Julianos in Sentinel helped me considerably. To Magister Oitos and his disciples, I here offer my sincere gratitude.

Nymphs are extremely shy creatures, no matter what the more obscene stories will tell you. No one who I've spoken with has had one seek him or her out. Thus to speak with a nymph requires energy and patience.

Out of courtesy for her privacy, I will not here give the location of the little grotto off the coast of Hammerfell where I found the nymph. It took three months of patient waiting, leaving presents where I knew the nymph would be, before the nymph stood still at my approach.

I remember I was carrying a bouquet of purple and white tetias, and she looked at them and then at me, and smiled. The effect of her smile was truly magical, I'm convinced. Her body was, of course, perfect; her face lovely and serene; her hair like silk flame. But until she smiled, she was beautiful in the abstract, a perfect statue by a master. The smile made her approachable and, thus, terrifying.

"For you," I said, attempting my first utterance of Nymph to a real nymph.

Her smile grew into a grin which became a giggle and then a laugh. The reader has doubtless heard of the silver laughter of the elves. The nymph's laugh is earthy and spontaneous, and very ... suggestive.

"And what do you want from me in return, mortal?" she asked.

"I am," There is no, I should say, known word in the Nymph language for scholar, "I am a man who likes to learn things. I want to learn things about you."

And I did.

Nymphs are the wisest, most wonderful creatures in Tamriel. My nymph, her name is Ayalea (a poor phonetic transcription of a word that sounds more like a light wind blowing through a small crack in a hollow chamber) and she knows more about the behavior and varieties of the deep woodland creatures than the greatest wood elf scholar I ever met. She taught me of flowers and ghosts and creatures too fast and timid to have ever been seen by man.

Ayalea taught me how to learn for the very first time. How to open my mind to all of the possibilities of life and how to use that knowledge, not just to hold in my cramped brain like a dragon's horde.

If you ever meet a nymph, speak to her.

* * *

Editor's note: the writer Vondham Barres is no longer a scholar at the Imperial University. He deposited this manuscript and disappeared from the civilized world. His current wherebouts are unknown.