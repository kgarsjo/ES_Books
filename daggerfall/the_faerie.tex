\cchapter{The Faerie}{Szun Triop}

\dropcap{F}{aerie} have been on Tamriel, in all probability, long before recorded history, perhaps since or before the days of the Elder Ones. The tales of their mischief are found in every culture, in most every village, town, and city-states in the Empire. Alternately they are called Faerie, Fey, Illyadi, Sprites, Pixies, and Sylphim, and their natures seem to flit from one story to the next with the same variation. It could almost be said that Faeries are anything unpredictable in nature.

The noted scholar Ahrtabazus studying at the time in the Crystal Tower of Sumurset Isle developed an interesting if controversial theory about Faerie. He organized the Fey variants on a chain, beginning with the glimmering sparks called Pixies or Whilloki by the Redguards at one end and the godlike beings such as Gheateus, Chonus, and Sygria at the other. In the middle are human and semi-human beings generating up to intelligent trees, brooks, rocks, even mountains. All of this was a new and completely original theory and would have prompted enthusiastic, if somewhat skeptical response had Ahrtabazus not added this footnote: "It may be that elves as a whole are part of this chain, above whilloki and below nephrine. They certainly have similar features and propensities for magicka as the other Faerie." (Ahrtabazus, "The Faerie Chain" Firsthold, 2E 456)

No elf liked to be put in a hierarchy slightly above whimsical pranksters like the whilloki, and Ahrtabazus was challenged on his assumptions based on very slight coincidences. Nevertheless, with modification, his Fairie Chain theory has gained wider and wider acceptance since its publication.

The hierarchial chain is not, in the strictest sense, an order of command. While Gheateus and Sygria are said to be surrounded by a host of minor Sylphim, faerie on the whole are not followers nor leaders. Their plans and schemes are not governed by a higher purpose, simply by their own whim.

To this most faerie scholars agree. Because it is based on coincidental evidence and supported by auxiliary theories, it may very well be wrong.